\documentclass{article}
\usepackage[utf8]{inputenc}
\usepackage[T2A]{fontenc}
\usepackage[russian]{babel}
\usepackage{graphicx}
\usepackage{amssymb}
\usepackage{amsthm}
\theoremstyle{plain}
\usepackage{mathtools}
\usepackage{varioref}
\usepackage[hidelinks]{hyperref}
\usepackage[nameinlink, capitalise, noabbrev]{cleveref}
\usepackage{tcolorbox}
\graphicspath{{./images/}}

\hypersetup{
    colorlinks,
    citecolor=black,
    filecolor=black,
    linkcolor=blue,
    urlcolor=black
}
\newcommand{\boxy}[2]{
    \begin{tcolorbox}[colback=black!5!white,colframe=black, title = {Кривые}]
        #2
    \end{tcolorbox}
}
\DeclarePairedDelimiterX\set[1]\lbrace\rbrace{\def\given{\;\delimsize\vert\;}#1}
\title{%
  Геометрия \\
  \large Домашняя контрольная работа № 2, вариант № 9}
\author{Максим Зыкин}
\date{27.11.2021}
\def\letus{%
    \mathord{\setbox0=\hbox{$\exists$}%
             \hbox{\kern 0.125\wd0%
                   \vbox to \ht0{%
                      \hrule width 0.75\wd0%
                      \vfill%
                      \hrule width 0.75\wd0}%
                   \vrule height \ht0%
                   \kern 0.125\wd0}%
           }%
}
\DeclareMathOperator{\rank}{rank}

\newcommand{\xdashrightarrow}[2][]{\ext@arrow 0359\rightarrowfill@@{#1}{#2}}
\newcommand{\xdashleftarrow}[2][]{\ext@arrow 3095\leftarrowfill@@{#1}{#2}}
\newcommand{\xdashleftrightarrow}[2][]{\ext@arrow 3359\leftrightarrowfill@@{#1}{#2}}
\def\rightarrowfill@@{\arrowfill@@\relax\relbar\rightarrow}
\def\leftarrowfill@@{\arrowfill@@\leftarrow\relbar\relax}
\def\leftrightarrowfill@@{\arrowfill@@\leftarrow\relbar\rightarrow}
\def\arrowfill@@#1#2#3#4{%
  $\math\thickmuskip\medmuskip\thickmuskip\thinmuskip\thickmuskip
   \relax#4#1
   \xleaders\hbox{$#4#2$}\hfill
   #3$%
}
\begin{document}
\maketitle
\tableofcontents
\vspace{3cm}
\boxy{}{
\begin{enumerate}

\item \textbf{Кривая 1}
$$F_1(x, y) = x^2 - 2xy + y^2 - 12x - 4y + 36 = 0$$
\item \textbf{Кривая 2}
$$
F_2(x, y) = 2x^2 + 5xy + 2y^2 - 6x - 3y - 8
$$ 
\end{enumerate}}
\newpage

\section{Задание № 1. Определение типа кривой.}
\begin{enumerate}
    \item \textbf{Кривая 1}

$$\delta = \Delta \begin{pmatrix} 
1 & -1 \\
-1 & 1
\end{pmatrix} = 0 \implies F_1(x, y)\; \text{--
кривая параболического типа}$$
Теперь определим её вид. Для этого посчитаем следующий инвариант:
$$ \Delta\begin{pmatrix}
    1 & -1 & -6 \\
    -1 & 1 & -2 \\
    -6 & -2 & 36\\
\end{pmatrix}  < 0 \implies \text{кривая не вырождена. Это парабола.}$$
$$-\tg^2{\alpha} + (1 - 1)\tg{\alpha} + 1 = 0
$$

$$
\left[ \begin{gathered}
    \tg{\alpha} = 1 \\
    \tg{\alpha} = -1
\end{gathered} \right.\implies \alpha = \frac{\pi}{4}\implies$$
\begin{equation}
    \begin{cases}
    \sin{\alpha} = \frac{\sqrt{2}}{2} \\
    \cos{\alpha} = \frac{\sqrt{2}}{2}
    \end{cases}
\end{equation}
\item \textbf{Кривая 2}

$$\delta = \Delta \begin{pmatrix} 
2 & \frac{5}{2} \\
\frac{5}{2} & 2
\end{pmatrix} = -\frac{9}{4} \implies F_2(x, y)\; \text{--
кривая гиперболического типа.}$$
Теперь определим, вырождена она или нет.\\
Для этого посчитаем следующий инвариант:
$$\Delta\begin{pmatrix}
    2 & \frac{5}{2} & -3 \\
    \frac{5}{2} & 2 & -\frac{3}{2} \\
    -3 & -\frac{3}{2} & -8\\
\end{pmatrix}  \ne 0 \implies \text{кривая не вырождена. Это гипербола.}$$
$$\frac{5}{2}\tg^2{\alpha} + (2 - 2)\tg{\alpha} - \frac{5}{2} = 0\; | \div \frac{5}{2}
$$
$$
\left[ \begin{gathered}
    \tg{\alpha} = 1 \\
    \tg{\alpha} = -1
\end{gathered} \right.\implies \alpha = \frac{\pi}{4}\implies$$
\begin{equation}
    \begin{cases}
    \sin{\alpha} = \frac{\sqrt{2}}{2} \\
    \cos{\alpha} = \frac{\sqrt{2}}{2}
    \end{cases}
\end{equation}
\end{enumerate}
\newpage
\section{Задание № 2. Приведение к каноническому типу.}
\begin{enumerate}
    \item \textbf{Кривая 1}
    \begin{equation}
        \begin{cases}
        x = x'\cos{\alpha} - y'\sin{\alpha}\\
        y = x'\cos{\alpha} + y'\sin{\alpha}
        \end{cases} = \;
        \begin{cases}
        x = \frac{x'}{\sqrt{2}} - \frac{y'}{\sqrt{2}}\\
        y = \frac{x'}{\sqrt{2}} + \frac{y'}{\sqrt{2}}
        \end{cases}
    \end{equation}
    $$F_1(x', y') = 2y'^2 - 8\sqrt{2}x' + 4\sqrt{2}y' + 36 = 0$$
    $$F_1(x', y') = 2y'^2 - 4\sqrt{2}x' + 2\sqrt{2}y' + 18 = 0 \;| + 2 - 2$$
    $$ (y' + \sqrt{2})^2 - 2 - 4\sqrt{2}x' + 18 = 0$$
    $$ (y' + \sqrt{2})^2 = 4\sqrt{2}(x' - \frac{4}{\sqrt{2}})$$
    \begin{equation}
        \begin{cases}
        x' = x'' + \frac{4}{\sqrt{2}} \\
        y' = y'' - \sqrt{2}
        \end{cases}
    \end{equation}
    Таким образом, каноническое уравнение параболы имеет вид: $$y''^2 = 4\sqrt{2}x''$$
    \item \textbf{Кривая 2}
    \begin{equation}
    \begin{cases}
     x = x'\cos{\alpha} - y'\sin{\alpha}\\
    y = x'\cos{\alpha} + y'\sin{\alpha}
     \end{cases} = \;
        \begin{cases}
        x = \frac{x'}{\sqrt{2}} - \frac{y'}{\sqrt{2}}\\
        y = \frac{x'}{\sqrt{2}} + \frac{y'}{\sqrt{2}}
        \end{cases}
    \end{equation}
    $$F_2(x', y') = x'^2 + y'^2 + 2x'y' + \frac{5}{2}(x'^2 + y'^2) + x'^2 + y'^2 - 2x'y' - \frac{6x'}{\sqrt{2}} - \frac{6y'}{\sqrt{2}} - \frac{3x'}{\sqrt{2}} + \frac{3y'}{\sqrt{2}} - 8 = 0\; |\, \cdot\, 2$$
    $$F_2(x', y') = 9x'^2 - y'^2 - 9\sqrt{2}x' - 3\sqrt{2}y' - 16 = 0\; | + \frac{9}{2} - \frac{9}{2}$$
    $$F_2(x', y') = (3x' - \frac{3}{\sqrt{2}})^2 - y'^2 - 3\sqrt{2}y' - \frac{9}{2} - 16 = 0$$
    $$F(x', y') = (3x' - \frac{3}{\sqrt{2}})^2 - (y' - \frac{3}{\sqrt{2}})^2 = 25 \;| \div 25$$
    $$\frac{ (x' - \frac{1}{\sqrt{2}})^2 }{(\frac{5}{3})^2} - \frac{ (y' - \frac{ 3 }{\sqrt{2}})^2}{5^2} = 1$$
    \begin{equation}
        \begin{cases}
        x' = x'' + \frac{1}{\sqrt{2}}\\
        y' = y'' + \frac{3}{\sqrt{2}}\\
        \end{cases}
    \end{equation}
    Таким образом, канонический вид гиперболы имеет вид:
    $$\frac{x''^2}{(\frac{5}{3})^2} - \frac{y''^2}{5^2} = 1$$
    \newpage
    \section{Задание № 3. Тип кривой и график.}
    Из канонического уравнения можно понять, что кривая № 1 -- это не что иное, как парабола.
    \begin{center}
    \includegraphics[ width=\textwidth]{images/geometry1parab.png}
    \captionof{График }{$F_1$}
    \end{center}
    \newpage
    Из канонического уравнения можно понять, что кривая № 2 -- это не что иное, как гипербола.
    \begin{center}
    \includegraphics[ width=\textwidth]{images/geometry1hyperb.jpg}
    \captionof{График }{$F_2$}
    \end{center}
    \vspace{3cm}
    \footnote{\large{См. \ref{pril} для графиков, нарисованных от руки.}}
    \newpage
    \section{Задание № 4. Характерные точки и прямые.}
    \textbf{Кривая 1}
    \begin{enumerate}
    \item В $O'x''y''$:\\
    \vspace{2mm}
    Фокус: $F(\frac{p}{2}, 0) = (\sqrt{2}, 0)$\\
    \vspace{2mm}
    Вершина: (0, 0)\\
    \vspace{2mm}
    Директриса: $x = -\frac{p}{2} = -\sqrt{2}$\\
    \vspace{2mm}
    Ось симметрии: $Ox$
    \item В исходной с. к.:\\
    Фокус: 
    $$x'' = x' - \frac{4}{\sqrt{2}}$$
    $$x' = \sqrt{2} + \frac{4}{\sqrt{2}} = \frac{6}{\sqrt{2}}$$
    $$y' = y'' - \sqrt{2} = -\sqrt{2}$$
    $$x = \frac{3\sqrt{2}}{\sqrt{2}} + 1 = 4$$
    $$y = 3 - 1 = 2$$
    $$(4; 2)$$
    Вершина:
    $$x = \frac{x'}{\sqrt{2}} - \frac{y'}{\sqrt{2}} = 2 + 1 = 3$$
    $$y = \frac{x'}{\sqrt{2}} + \frac{y'}{\sqrt{2}} = 2 - 1 = 1 $$
    $$(3; 1)$$
    Точка пересечения с $OX$:
    $$x^2 - 12x + 36 = 0$$
    $$D = 144 - 144 = 0\implies \text{ парабола касается оси абсцисс.}$$
    $$x = 6$$
    Точка пересечения с $Oy$:
    $$y^2 - 4y + 36 = 0$$
    $$y_{1,2} = \frac{4 \pm \sqrt{16 - 144}}{2}, $$
    $$ D < 0 \implies \emptyset  \text{ пересечений с осью ординат.}$$
    Директриса: 
    $$x'' = -\sqrt{2}\implies x' = \frac{4}{\sqrt{2}} - \sqrt{2} = \sqrt{2}$$
    $$\begin{pmatrix}
        x'\,\\
        y'\;(= 0)
    \end{pmatrix} =  \begin{pmatrix}
        \cos{\alpha} & \sin{\alpha} \\
        -\sin{\alpha} & \cos{\alpha}
    \end{pmatrix}\begin{pmatrix}
        x \\ 
        y
    \end{pmatrix}$$
    $$\sqrt{2} = \frac{x}{\sqrt{2}} + \frac{y}{\sqrt{2}} \implies y = 2 - x$$
    Ось симметрии:
    $$\frac{x - 4}{-1} = \frac{y - 2}{-1}$$
    $$2 - y = 4 - x$$
    $$y = x - 2$$
    \end{enumerate}
    \textbf{Кривая 2}
    \begin{enumerate}
        \item В $O'x''y''$:\\
        Асимптоты: $y = \pm \frac{bx}{a} = \pm 3x$\\
        \vspace{2mm}
        Эксцентриситет: $c = \sqrt{a^2 + b^2} = 5\frac{\sqrt{10}}{3}$
        $$\varepsilon = \frac{c}{a} = \sqrt{10}$$
        \vspace{2mm}
        Директрисы: $x = \mp \frac{a}{\varepsilon} = \mp \frac{5}{3\sqrt{10}}$\\
        \vspace{2mm}
        Фокусы: $(\mp c; 0) = (\mp 5\frac{\sqrt{10}}{3}; 0)$\\
        \vspace{2mm}
        Вершины: $(\mp \frac{5}{3}; 0)$\\
        \item В исходной с. к.:\\
        Фокусы:
        $$x' = x'' + \frac{1}{\sqrt{2}} \implies x' = \frac{\mp 10\sqrt{5} + 3}{3\sqrt{2}}$$
        $$y' = y'' + \frac{3}{\sqrt{2}} \implies y' = \frac{3}{\sqrt{2}}$$
        $$ x = \frac{\mp 10\sqrt{5} - 6}{6} $$
        $$y = \frac{(\mp 10\sqrt{5} + 3 + 9}{6}$$
        $$(\frac{\mp10\sqrt{5} - 6}{6}; \frac{\mp 10\sqrt{5} + 12}{6})$$
        Вершины:
        $$ x'' = \mp \frac{5}{3}$$
        $$ x' = \frac{\mp5\sqrt{2} + 3}{3\sqrt{2}}$$
        $$y' = \frac{3}{\sqrt{2}}$$
        Аналогично поиску фокусов:
        $$x = \frac{\mp5\sqrt{2} - 6}{6}$$
        $$y = \frac{\mp 5\sqrt{2} + 12}{6}$$
        $$( \frac{\mp5\sqrt{2} - 6}{6}; \frac{\mp 5\sqrt{2} + 12}{6})$$
        Асимптоты:
        $$y'' = \pm 3x$$
        $$(y' - \frac{3}{\sqrt{2}} = \pm 3(x' - \frac{1}{\sqrt{2}})$$
        $$-\sin{\alpha}x + \cos{\alpha}y - \frac{3}{\sqrt{2}} = \pm 3( \cos{\alpha}x + \sin{\alpha}y - \frac{1}{\sqrt{2}})$$
        $$-\frac{x}{\sqrt{2}} + \frac{y}{\sqrt{2}} - \frac{3}{\sqrt{2}} = \pm \frac{3x}{\sqrt{2}} \pm \frac{3y}{\sqrt{2}} \mp \frac{3}{\sqrt{2}}$$
        $$ -x + y - 3 \mp 3x \mp 3y \pm 3 = 0$$
        Асимптота 1:
        $$y = -2x$$
        Асимптота 2:
        $$ y = -\frac{x}{2} + \frac{3}{2} $$
        Директрисы:
        $$ x'' = \mp \frac{5}{3\sqrt{10}} $$
        $$x' = \frac{\mp 5\sqrt{2} + 3\sqrt{10}}{6\sqrt{5}}$$
        $$x' = x\cos{\alpha} + y\sin{\alpha}$$
        $$\frac{x}{\sqrt{2}} + \frac{y}{\sqrt{2}} = \frac{\mp 5\sqrt{2} + 3\sqrt{10}}{6\sqrt{5}}$$
        $$y = \mp \frac{10}{6\sqrt{5}} + 1 - x$$
        Точки пересечения с $Ox$:
        $$ x^2 - 3x - 4 = 0$$
        По т. Виетта:
        $$x_{1, 2} = 4; -1$$
        $$(4; 0), (-1; 0) $$
        Точки пересечения с $Oy$:
        $$ 2y^2 - 3y - 8 = 0$$
        $$y_{1, 2} = \frac{3 \pm \sqrt{73}}{4}$$ 
        $$(0; \frac{3 + \sqrt{73}}{4}), (0; \frac{3 - \sqrt{73}}{4})  $$
        Центр: 
        $$ x = \frac{1}{2} - \frac{3}{2} = -1$$
        $$y = \frac{1}{2} + \frac{3}{2} = 2$$
        $$(-1; 2)$$
    \end{enumerate}
    
\end{enumerate}
\newpage
\section{Приложение: графики, нарисованные от руки.}\label{pril}
\begin{center}
    \includegraphics[ width=\textwidth]{parabola.jpg}
    \captionof{График }{$F_1$}
    \end{center}

\begin{center}
    \includegraphics[ width=\textwidth]{hyperbola.jpg}
    \captionof{График }{$F_2$}
    \end{center}


\newpage
\section{Список литературы.}
\begin{enumerate}
    \item Клетеник, Д. В. Сборник задач по аналитической геометрии / Д. В. Клетеник. — 13-е изд. — Москва : Наука, 1980. — 243 c. — Текст : непосредственный.
\end{enumerate}
\end{document}
