\documentclass{article}
\usepackage[utf8]{inputenc}
\usepackage[T2A]{fontenc}
\usepackage[russian]{babel}
\usepackage{graphicx}
\usepackage{amssymb}
\usepackage{amsthm}
\theoremstyle{plain}
\usepackage{mathtools}
\DeclarePairedDelimiterX\set[1]\lbrace\rbrace{\def\given{\;\delimsize\vert\;}#1}
\title{%
  Геометрия \\
  \large Домашняя контрольная работа № 1, вариант № 9}
\author{Максим Зыкин}
\date{25.10.2021}
\def\letus{%
    \mathord{\setbox0=\hbox{$\exists$}%
             \hbox{\kern 0.125\wd0%
                   \vbox to \ht0{%
                      \hrule width 0.75\wd0%
                      \vfill%
                      \hrule width 0.75\wd0}%
                   \vrule height \ht0%
                   \kern 0.125\wd0}%
           }%
}
\DeclareMathOperator{\rank}{rank}

\newcommand{\xdashrightarrow}[2][]{\ext@arrow 0359\rightarrowfill@@{#1}{#2}}
\newcommand{\xdashleftarrow}[2][]{\ext@arrow 3095\leftarrowfill@@{#1}{#2}}
\newcommand{\xdashleftrightarrow}[2][]{\ext@arrow 3359\leftrightarrowfill@@{#1}{#2}}
\def\rightarrowfill@@{\arrowfill@@\relax\relbar\rightarrow}
\def\leftarrowfill@@{\arrowfill@@\leftarrow\relbar\relax}
\def\leftrightarrowfill@@{\arrowfill@@\leftarrow\relbar\rightarrow}
\def\arrowfill@@#1#2#3#4{%
  $\math\thickmuskip\medmuskip\thickmuskip\thinmuskip\thickmuskip
   \relax#4#1
   \xleaders\hbox{$#4#2$}\hfill
   #3$%
}
\begin{document}

\maketitle
\begin{enumerate}
\item \textbf{Задача 1}
$$A(-5; 0),\; B(4; -2),\; C(-2; -6)$$
$M$ -- середина $BC$. $M (1; -4)$
$$\overline{AM}(6; -4) \implies$$
Ответ:
$$\frac{x + 5}{6} = \frac{y}{-4}$$
\item \textbf{Задача 2}
$$M(-9; -2)$$
$$l: x + y + 1 = 0$$
$$\overline{n_{l}}(1; 1)$$
$$ M_l = M + \alpha\overline{n_l},\; \alpha \in \mathbb{R} \implies M_l (-9 + \alpha; -2 + \alpha)$$
$$\alpha: l(M_l) = -9 + \alpha - 2 + \alpha + 1 = 0 \implies \alpha = 5 $$
$$M_{sym} = M_l + \alpha\overline{n_l} = (-4; 3) + (5; 5) = (1; 8)$$
Ответ:
$$(1; 8)$$
\item \textbf{Задача 3}
$$A(-4; 2), B(4; 0), C(5; 4)$$
$$HM : \frac{x - x_B}{n_{AC_x}} = \frac{y - y_B}{n_{AC_y}},\; M = pr^{H}_{AC}$$
$$AC : \frac{x + 4}{9} = \frac{y - 2}{2}\implies \overline{n_{AC}}(-2; 9)$$
$$HM : \frac{x - 4}{-2} = \frac{y}{9}$$
Ответ:
$$\frac{x - 4}{-2} = \frac{y}{9}$$
\item \textbf{Задача 4}
$$A(1; -6)$$
$$l : x - y - 1 = 0$$
$$ A \not\in l \implies AA_l \perp l$$
$$\overline{n_l}(1; -1) \implies A_l \in l\; A_l = A + \alpha\overline{n_l} = (1; -6) + (\alpha; -\alpha)$$
$$\alpha: l(A_l) = 1 + \alpha + 6 + \alpha - 1 = 0 \implies \alpha = -3$$
$$S = |AA_l|^2 = \sqrt{(-2 - 1)^2 + (-9 + 6)^2}^2 = 18$$
Ответ: 
$$S = 18$$
\item \textbf{Задача 5}
$$A(-3; 3), B(11; -5), C(-9; 3)$$
$$\letus\, M \in AB: AM = MB,\; \letus\, N \in AC: AN = NC$$
$$M(4; -1),\; N(-6; 3) \implies l : \frac{x - 4}{-10} = \frac{y + 1}{4} $$
Ответ:
$$\frac{x - 4}{-10} = \frac{y + 1}{4}$$
\item \textbf{Задача 6}
$$A(-3; 5), B(18; -7), C(7; 5), D(-2; -7)$$
$$AB : \frac{x + 3}{21} = \frac{y - 5}{-12}\implies 4x + 7y - 23 = 0$$
$$CD : \frac{x - 7}{-9} = \frac{y - 5}{-12}\implies 4x - 3y - 13 = 0$$
\begin{equation}
    \begin{cases}
    4x + 7y - 23 = 0\\
    4x - 3y - 13 = 0
    \end {cases}
    = \hspace{2mm}
    \begin{cases}
    10y = 10\\
    4x - 3y - 13 = 0
    \end{cases}
    = \hspace{2mm}
    \begin{cases}
    y = 1\\
    x = 4
    \end{cases}
\end{equation}

Ответ:
$$(4; 1)$$
\item \textbf{Задача 7}
$$P(-3; -6; 17)$$
$$\pi: 3x + 2y - 5z + 30 = 0$$
$$\overline{n_\pi}(3; 2; -5)$$
$$P_\pi = P + \alpha\overline{n_\pi} = (-3; -6; 17) + (3\alpha; 2\alpha; -5\alpha)$$
$$\alpha: \pi(P_\pi) = 3(-3 + 3\alpha) + 2(-6 + 2\alpha) - 5(17 - 5\alpha) + 30 = 0 \implies \alpha = 2$$
$$P_{sym} = P_\pi + \alpha\overline{n_\pi} = (3; -2; 7) + (6; 4; -10) = (9; 2; -3)$$
Ответ:
$$(9; 2; -3)$$
\item \textbf{Задача 8}
$$\alpha: 4x + 4y - 7z +20 = 0,\; M\,(0; y_0; 0),\; \rho = 8$$
$$\rho(M, \alpha) = \frac{|Ax_M + By_M + Cz_M + D|}{\sqrt{A^2 + B^2 + C^2}} = \frac{|4y +  20|}{\sqrt{81}} = 8$$
$$ \left[ \begin{gathered}
    -4y - 20 = 72\\
    4y + 20 = 72
\end{gathered} 
\right.$$
Ответ: 
$$(0; -23; 0),\; (0; 13; 0)$$

\item \textbf{Задача 9}
$$\xi : 2x + 2y - 5z + 10 = 0,\; \psi : 4x - y + 4z - 5 = 0$$
$$\frac{|2x + 2y - 5z + 10|}{\sqrt{33}} = \frac{|4x - y + 4z - 5|}{\sqrt{33}}$$
$$|2x + 2y - 5z + 10| = |4x - y + 4z - 5|$$
$$\left[ 
    \begin{gathered}
    2x + 2y - 5z + 10 = 4x - y + 4z - 5 \\
    -2x - 2y + 5z - 10 = 4x - y + 4z - 5
    \end{gathered} \right.
\hspace{6mm}    = \hspace{6mm}
\left [
    \begin{gathered}
    -2x + 3y - 9z + 15 = 0 \\
    -6x - y + z - 5 = 0
    \end{gathered}
    \right.$$
Ответ:
$$
    \begin{gathered}
    -2x + 3y - 9z + 15 = 0 \\
    -6x - y + z - 5 = 0
    \end{gathered}
  $$

\item \textbf{Задача 10}
$$l : \frac{x}{1} = \frac{y + 1}{1} = \frac{z - 3}{0}$$
$$\pi : A(4; 1; -6), B(2; 0; -4), C(3; 0; -6) \in \pi$$
$$\det
    \begin {pmatrix}
    x - 4 & -1 & -2 \\
    y - 1 & -1 & -1 \\
    z + 6 & 0 & 2 
    \end{pmatrix}
    = -2x - 2y + z + 12 = 0 \leftarrow \pi
    $$
    $$ \overline{n_\pi} (-2; -2; 1),\; \overline{p_l} = (1; 1; 0)$$
    $$ (\overline{n_\pi}, \overline{p_l}) = -2 - 2 = -4 \implies$$
    Ответ: \\
    Прямая пересекает плоскость в некой точке $M$.
    \item \textbf{Задача 11}
    \begin{equation}
    l : 
        \begin{cases}
        3x + 2y - 5z - 26 = 0 \\
        4x + 2y - 7z  - 36 = 0
        \end{cases}
        \hspace{1cm}
        l_1 : 
        \begin{cases}
        x = 5 - 4t \\
        y = -5 + t \\
        z = 1 - 2t
        \end{cases}
    \end{equation}
    $$\overline{p_l} = [\overline{n_1}, \overline{n_2}] = \det \begin {pmatrix}
    i & j & k \\
    3 & 2 & -5 \\
    4 & 2 & -7
    \end{pmatrix} = -4i - (-1)j + (-2)k = (-4; 1; -2)$$
    Подберём общую точку (которая будет принадлежать прямой $l$).
    $$P(2; 0; -4) \in l\implies l : \frac{x - 2}{-4} = \frac{y}{1} = \frac{z + 4}{-2}$$
    $$l_1 : \frac{x - 5}{-4} = \frac{y + 5}{1} = \frac{z - 1}{-2}$$
    Видно, что направляющие вектора совпадают. 
    $$\overline{M}(x_2 - x_1; y_2 - y_1; z_2 - z_1) = (3; -5; 5)$$
    $$\rank \begin{pmatrix}
        3 & -5 & 5\\
        -4 & 1 & -2\\
        -4 & 1 & -2
    \end{pmatrix} = 2 \implies l \parallel l_1$$
    Ответ:
    $$l \parallel l_1 $$

\item \textbf{Задача 12}
\begin{equation}
l:
    \begin{cases}
    y - z - 1 = 0\\
    4x - y - z + 1 = 0
    \end{cases}
\end{equation}
$$ l_1 : \frac{x - 2}{1} = \frac{y - 2}{2} = \frac{z + 2}{2} $$
$$P \in l_1, P(2; 2; -2)$$
Сведём задачу к нахождению расстояния от точки до прямой.
$$\overline{p_l} = [\overline{n_1}, \overline{n_2}] = \det\begin{pmatrix}
    i & j & k \\
    0 & 1 & -1 \\
    4 & -1 & -1 
\end{pmatrix} = -2i - 4j -4k = (-2; -4; -4)$$
$$M \in l\, M(0; -1; 2)$$
$$\rho(l, M) = \frac{[\overline{MP}, (1, 2, 2)]}{\sqrt{1 + 4 + 4}}$$
$$\overline{MP}(2; 3; -4) $$
$$ \det \begin{pmatrix} 
    i & j & k \\
    2 & 3 & -4 \\
    1 & 2 & 2 
\end{pmatrix}  = 14i - 8j + k = (14; -8; 1)$$

$$\rho(l, M) = \frac{\sqrt{261}}{3} = \sqrt{29} $$
Ответ:
$$\sqrt{29}$$
\item \textbf{Задача 13}
\begin{equation}
l:
    \begin{cases}
    4x + 7y + z + 31 = 0\\
    5x + 8y + z + 34 = 0
    \end{cases} 
\end{equation}
$$l_1 : \frac{x}{1} = \frac{y + 5}{-7} = \frac{z - 4}{3}$$
$$\overline{p_l} = [\overline{n_1}, \overline{n_2}] = \det\begin{pmatrix}i & j & k \\ 4 & 7 & 1 \\ 5 & 8 & 1\end{pmatrix} = -i + j -3k = (-1; 1; -3)$$
$$\overline{p_{l_1}} (1; -7; 3)$$
$$\cos{\angle{(\overline{p_{l_1}}, \overline{p_l})}} = \frac{-1 - 7 - 9}{\sqrt{1 + 1 + 9}\sqrt{1 + 9 + 49}} = \frac{-17}{\sqrt{649}}$$
Ответ:
$$\arccos{(\frac{-17}{\sqrt{649}})}$$
\item \textbf{Задача 15}
$$P(10; -15; -22)$$
$$l: \frac{x - 6}{3} = \frac{y - 1}{-2} = \frac{z - 2}{7} = t $$
$$M(6; 1; 2),\;\overline{p}(3; -2; 7)$$
$$\xi: 3x - 2y + 7z + D = 0$$
$$\xi(P) = 30 + 30 - 154 = -94 = -D \implies D = 94 $$
\begin{equation}
\begin{cases}
x = 6 + 3t\\
y = 1 - 2t \\
z = 2 + 7t
\end{cases}
\end{equation}
$$\xi(x, y, z) = 3 (6 + 3t) - 2(1 - 2t) + 7(2 + 7t) + 94 = 0 \implies t = -2 $$ 
$$M_l(0; 5; -12)$$
$$ P_{sym} = P(10; -15; -22) + 2\overline{PM_l}(-10; 20; 10) = (-10; 25; -2)$$
Ответ:
$$(-10; 25; -2)$$

\item \textbf{Задача 16}
$$P(-1; 24; 8)$$
$$\pi : x - 8y - 3z + 69 = 0$$
$$\overline{n_\pi}(1; -8; -3)$$
$$P_\pi = P + \alpha \overline{n_\pi}$$
$$\pi(P_\pi) = -1 + \alpha - 8(24 - 8\alpha) - 3(8 - 3\alpha) + 69 = 0 \implies \alpha = 2$$
$$P_\pi (-1; 24; 8) + (2; -16; -6) = (1; 8; 2)$$
Ответ:
$$(1; 8; 2) $$

\item \textbf{Задача 17}
$$P(5; -38; 8)$$
$$l: \frac{x - 4}{4} = \frac{y - 4}{-3} = \frac{z - 1}{-2}$$
Плоскость, проходящая через точку $P$ -- $\psi$
$$ \psi: 4x - 3y - 2z + D = 0 \implies \psi(P) = 20 + 114 - 16 + D = 118 + D = 0 \implies D = -118 $$
$$ \psi : 4x - 3y - 2z - 118 = 0 $$
\begin{equation}
    \begin{cases}
    x = 4t + 4 \\
    y = 4 - 3t \\
    z = 1 - 2t
    \end{cases}
\end{equation}
$$\psi(x, y, z) = 16t + 16 - 12 + 9t - 2 + 4t - 118 = 0\implies t = 4 $$
Ответ:
$$(20; -8; -7)$$

\item \textbf{Задача 18}
\begin{equation}
l:
    \begin{cases}
    x - 8y - 3z - 54 = 0 \\
    2x - 9y - 4z - 64 = 0
    \end{cases}
\end{equation}
$$ \pi: 3x - 10y - 5z - 60 = 0 $$
$$\overline{p_l} = [\overline{n_1}, \overline{n_2}] = \det\begin{pmatrix} i & j & k\\ 1 & -8 & -3 \\ 2 & -9 & -4 \end{pmatrix} = 5i - 2j + 7k = (5; -2; 7) $$
Найдём точку, которая удовлетворяет системе методом Gauss'a:
$$\begin{pmatrix}
    1 & -8 & -3 & 54 \\
    2 & -9 & -4 & 60
    \end{pmatrix}$$
    = 
    \begin{align}
    x = \frac{5z}{7} - \frac{6}{7}\\
    y = -\frac{2z}{7} - \frac{48}{7}
    \end{align} 

$$P(\frac{29}{7}; -\frac{62}{7}; 7)$$
$$ l : \frac{x - \frac{29}{7}}{5} = \frac{y + \frac{62}{7}}{-2} = \frac{z - 7}{7} = t $$
\begin{equation}
    \begin{cases}
    x = 5t + \frac{29}{7}\\
    y = -2t - \frac{62}{7}\\
    z = 7t + 7
    \end{cases}
\end{equation}
$$\pi(x, y, z) = 3(5t - \frac{29}{7}) - 10(-2t - \frac{62}{7}) - 5(7t + 7) - 60 = 0\implies 6 \neq 0\implies t = \emptyset \implies l \not\in \pi \; (*)$$
(*) и то, что $(\overline{n}(A, B, C), \overline{p_l}) = 15 + 25 - 35 = 0 \;(\implies \overline{n_\pi} \perp \overline{p_l})$
$$\implies l \parallel \pi $$
Ответ: 
$$ l \parallel \pi$$
\item \textbf{Задача 14}
$$l: \frac{x + 4}{8} = \frac{y + 1}{3} = \frac{z + 12}{8}$$
$$m: \frac{x - 5}{24} = \frac{y - 10}{5} = \frac{z + 8}{16}$$
$$\overline{p_l} (8; 3; 8),\; \overline{p_m}(24; 5; 16)$$
Сразу же замечаем, что направляющие вектора не пропорциональны и, следовательно, не коллинеарны (а значит, что и прямые не параллельны). Теперь узнаем, компланарны они или нет.
$$ P_l(-4; -1; -12),\; P_m(5; 10; -8), \overline{P_lP_m}(9; 11; 4)$$
$$ \det\begin{pmatrix}
    9 & 11 & 4\\
    8 & 3 & 8 \\
    24 & 5 & 16
\end{pmatrix}  = 72 + 704 - 128 \neq 0 \implies $$
Направляющие вектора и вектор $\overline{P_lP_m}$ не компланарны, следовательно, прямые скрещиваются. 
Можно найти расстояние, сведя задачу к поиску расстояния от точки до плоскости. Примем, что мы ищем расстояние до точки на $m$ от плоскости $\pi$, в котрой лежит прямая $l$.
$$ \overline{n_{lm}} = \det\begin{pmatrix}
    i & j & k \\
    8 & 3 & 8 \\
    24 & 5 & 16 
\end{pmatrix}  = 8i + 64j - 32k \implies \overline{n_{lm}}(8; 64; -32)$$
$\letus\,\xi$ -- плоскость, в которой лежит $l$. Найдём её уравнение. 
$$ 8x + 64y - 32z + D = 0 \implies D = -8 (-4) + (-64)(-1) + 32(-12) = 32 + 64 - 384 = -288  $$
$$ \xi: 8x + 64y - 32z - 288 = 0 \implies = x + 8 y - 4z - 36 = 0$$
$$ \rho(\xi, P_m) = \frac{|Ax_0 + By_0 + Cz_0 + d|}{\sqrt{A^2 + B^2 + C^2}} = \frac{5 + 8 * 10 + 32 - 36}{\sqrt{81}} = 9 $$

Ответ:
$$9$$
\end{enumerate}

\end{document}
